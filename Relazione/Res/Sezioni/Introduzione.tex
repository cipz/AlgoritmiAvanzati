\section{Introduzione}

Questo elaborato ha lo scopo di illustrare il lavoro svolto per il primo homework del corso \textit{Algotitmi Avanzati}.\\
L'homework ha come scopo quello di confrontare tra loro gli algoritmi per il problema del \textit{Minimum Spanning Tree} (o \textit{MST}) visti a lezione, questi sono:
\begin{enumerate}
	\item \textit{Algoritmo di Prim implementato con Heap} (Sez.~\ref{prim})
	\item \textit{Algoritmo di Kruskal nella sua implementazione ``naive'' di complessità $O(mn)$} (Sez.~\ref{kruskal_naive})
	\item \textit{Algoritmo di Kruskal implementato con Union-Find} (Sez.~\ref{kruskal_uf})
\end{enumerate}

Il linguaggio in cui sono stati implementati questi algoritmi è \texttt{Python}.
Abbiamo deciso di utilizzare questo al contrario di altri linguaggi come \textit{C++} o \textit{Java} in quanto questa scelta ci ha permesso di utilizzare i \textit{Jupyter Notebook} e di programmare utilizzando l'IDE \textit{PyCharm} oppure con \textit{Google Colab}.

Nella Sez.~\ref{risultati} sono presenti i risultati degli MST calcolati sul dataset dato, contenente 68 grafi di esempio, di dimensione compresa tra $10$ e $100000$ vertici e descritti in file \texttt{.txt}.
Sono inoltre illustrati i grafici delle performance dei vari algoritmi.

Il lavoro è stato suddiviso equamente tra i membri del gruppo nel seguente modo:
\begin{itemize}
	\item Algoritmo di Prim: Nicolò Tartaggia
	\item Algoritmo di Kruskal ``naive'': Lorenzo Busin
	\item Algoritmo di Kruskal con Union Find: Ciprian Voinea
\end{itemize}

\pagebreak
