\section{Conclusione}
Una volta raccolte le informazioni relative alle performance dei tre algoritmi eseguiti nel dataset di input queste sono state collezionate per costruire un grafico che li mettesse a confronto.
\begin{figure}[H]
	\hspace{-1cm}\includegraphics[width=19cm]{Img/compare.png}
	\caption{Confronto tra le performance dei tre algoritmi}
\end{figure}
 Dalla figura 4 è possibile osservare che per grafi di piccole dimensioni, quindi con un numero di vertici circa pari o inferiore a 1000, la differenza di prestazioni dei diversi algoritmi non è molto evidente. Quando le dimensioni dei grafi di input crescono i tempi di esecuzione dell'algoritmo di Prim con implementazione "naive" crescono in maniera esponenziale fino ad impiegare svariati minuti per effettuare il calcolo del MST per i grafi più grandi presenti nel dataset. Gli altri due algoritmi, Prim e Kruskal con Union find, hanno tempi di esecuzione più simili tra loro: pochi secondi per il calcolo del MST anche su grafi di grandi dimensioni. \\
Questi risultati sono coerenti con le complessità degli algoritmi, infatti quello con la complessità più alta è Prim in versione "naive" O(mn). Gli altri due, invece, presentano la stessa complessità algoritmica sebbene sia possibile affermare che l'algoritmo di Prim implementato con Union find presenti dei tempi di esecuzione inferiori rispetto a Prim.
\pagebreak