\section{Algoritmo di Karger}\label{karger}

\begin{lstlisting}[mathescape=true]
Karger(G,k)
	select a vertex r $\in$ G.V to be a root vertex
	compute a MST T for G from root r using MST-PRIM(G, c, r)
	let H be a list of vertices, ordered according to when they are first visited in a preordered tree walk of T
	return the hamiltonian cycle H
	
Full contraption()
	
\end{lstlisting}	

L'algoritmo di 2-approssimazione utilizza come sottoprocedura l'algoritmo di Prim, che fornisce un MST il cui peso dà un lower bound sulla lunghezza del ciclo della soluzione ottima del TSP sul grafo dato.\\
Questo algoritmo sfrutta la disuguaglianza triangolare, ovvero dati tre nodi A, B, C, la distanza tra A e C può al massimo la distanza tra A e B sommata alla distanza tra B e C.\\
Fintanto che la funzione del costo soddisfa tale regola, viene utilizzato il minimum spanning tree per creare un ciclo con un costo che è massimo due volte il costo della soluzione ottima, per questo si dice 2-approssimato.

\subsection{Strutture dati}

	A differenza degli altri homework, per migliorare le performance dell'algoritmo in questione, abbiamo deciso di utilizzare solamente la struttura dati Graph, implementata con i seguenti campi e metodi:
	\begin{itemize}
		\item \texttt{\textbf{n\_nodes}}
		\item \texttt{\textbf{n\_edges}}
		\item \texttt{\textbf{nodes}}
		\item \texttt{\textbf{edges}}
		\item \texttt{\textbf{init}}
		\item \texttt{\textbf{addEdge}}
		\item \texttt{\textbf{buildGraph}}
	\end{itemize}
	
\subsection{Funzioni}
	
	Le funzioni ausiliarie utilizzate per calcolare la soluzione dell'algoritmo di Karger sono le seguenti:
	\begin{itemize}
		\item \texttt{\textbf{readInput}}: ;
	\end{itemize}

\subsection{Implementazione}
	
	La soluzione del problema è stata implementata nel seguente modo:
	\begin{itemize}
		\item viene costruito il grafo \texttt{g} a partire dai lista di vertici letta dal file in input;
		\item si utilizza l'algoritmo di Prim per calcolare l'MST del grafo appena creato;
		\item tale MST viene convertito, tramite l'uso della funzione \texttt{mst\_to\_tree}, in una lista che rappresenta i padri di ciascun nodo;
		\item andando a riordinare questa lista tramite la funzione \texttt{preorder}, si ottiene un ciclo hamiltoniano al quale viene appeso in coda il primo nodo, con indice 0, per chiuderlo;
		\item viene quindi calcolata la soluzione finale andando a sommare i pesi degli archi che collegano i nodi del ciclo.
	\end{itemize}
		
\subsection{Complessità}

	Per calcolare la complessità di questo algoritmo bisogna tenere in considerazione anche le operazioni che vengono svolte da Prim e, di conseguenza, quelle svolte sullo heap.\\
	La complessità totale è quindi $O(|V|^2)$, dove V indica la lista di vertici del grafo.

\pagebreak